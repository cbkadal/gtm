% REV03 Sat 07 Aug 2021 11:40:10 WIB
% REV02 Fri 06 Aug 2021 11:09:59 WIB
% REV01 Thu 05 Aug 2021 14:26:40 WIB
% START Tue 03 Aug 2021 13:44:15 WIB

\chapter{Remodeling Grounded Theory}
\section*{\citeauthor{article.glaser04}}
The Grounded Theory Review, Vol. 04, No.1, Nov 2004.
\section*{Abstract}
\addcontentsline{toc}{section}{Abstract}

This paper outlines my concerns with Qualitative Data Analysis' (QDA)
numerous remodelings of Grounded Theory (GT) and the subsequent eroding
impact. I cite several examples of the erosion and summarize essential
elements of classic GT methodology. It is hoped that the article will clarify my
concerns with the continuing enthusiasm but misunderstood embrace of GT by
QDA methodologists and serve as a preliminary guide to novice researchers
who wish to explore the fundamental principles of GT.

\section*{Introduction}

The difference between the particularistic, routine, normative data we all garner
in our everyday lives and scientific data is that a methodology produces the latter.
This is what makes it scientific.
This may sound trite, but it is just the beginning of many complex issues.
Whatever methodology may be chosen to make an ensuing research
scientific has many implicit and explicit problems.
It implies a certain type of data collection,
the pacing, and timing for data collection, a type of analysis,
and a specific type of research product.

In the case of qualitative data, the explicit goal is the description.
The particular issue articulated in much of the literature regarding
qualitative data analysis (QDA) methodology is the data's accuracy,
truth, trustworthiness, or objectivity.
This worrisome accuracy of the data focuses on its subjectivity, interpretative nature,
plausibility, data voice, and constructivism.
Achieving accuracy is always problematic with a QDA methodology.

These are a few of the problems of description. Other QDA problems include
pacing of data collection, the volume of data, the procedure and rigor of data
analysis, generalizability of the unit findings, the framing of the ensuing analysis, and the product. These issues and others are debated at length in the
qualitative research literature. The worrisome accuracy of qualitative data description
continually concerns qualitative researchers and their audiences. I have
addressed these problems at length in "The Grounded Theory Perspective:
Conceptualization Contrasted with Description"
\citep{book.glaser01}.







\begin{quote}
''Today’s general textbooks perpetuate the established marketing
management epic from the 1960s with the new just added as extras. It
is further my contention that marketing education has taken an
unfortunate direction and has crossed the fine line between education
and brainwashing. The countdown of a painful—but revitalizing—
process of deprogramming has to be initiated.

What do we need in such a situation? A shrink? No, it is less
sophisticated than that. All we need is systematic application of
common sense, both in academe and in corporations.We need to use
our observational capacity in an inductive mode and allow it to receive
the true story of life, search for patterns and build theory.Yes, theory.
General marketing theory that helps us put events and activities into a
context. This is all within the spirit of grounded theory, wide spread in
sociology but little understood by marketers. My interpretation of a
recent book on the subject by 
\citep{book.glaser01}
is as follows: ‘take the
elevator from the ground floor of raw substantive data and description to
the penthouse of conceptualization and general theory. And do this
without paying homage to the legacy of extant theory.’ In doing this,
complexity, fuzziness and ambiguity are received with cheers by the
researchers and not shunned as unorderly and threatening as they are
by quantitative researchers. Good theory is useful for scholars and
practicing managers alike \citep[p. 132]{article.gummensson02}.''
\end{quote}

\section*{Copyright}
\addcontentsline{toc}{section}{Copyright}

As an academic, open access journal,
the Grounded Theory Review is dedicated to the open exchange of information.
Copies of articles in this journal may be distributed for noncommercial purposes
free of charge and without permission.
The copyright of papers published in the Grounded Theory Review stay with the author.
The Grounded Theory Review does not have a specific creative commons license,
but allows readers to read, download, copy, distribute, print, search,
or link to the full texts of its articles.  Derivative works are allowed with proper citation.


\nocite{article.cynthia10}
\nocite{article.lowe97}
\nocite{article.morse95}
\nocite{book.creswell98}
\nocite{book.glaser65}
\nocite{book.glaser67}
\nocite{book.glaser78}
\nocite{book.glaser92}
\nocite{book.glaser93}
\nocite{book.glaser94}
\nocite{book.glaser95}
\nocite{book.glaser98a}
\nocite{book.glaser98b}
\nocite{incollection.may94}
\nocite{incollection.morse94}


\bibliographystyle{apalikerd}
\bibliography{bib}

