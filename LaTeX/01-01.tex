% REV06 Tue 10 Aug 2021 10:31:54 WIB
% REV05 Mon 09 Aug 2021 08:57:31 WIB
% REV04 Sun 08 Aug 2021 09:13:19 WIB
% REV03 Sat 07 Aug 2021 11:40:10 WIB
% REV02 Fri 06 Aug 2021 11:09:59 WIB
% START Tue 03 Aug 2021 13:44:15 WIB

\chapter{Remodeling Grounded Theory}
\section*{\citeauthor{article.glaser04}}
The Grounded Theory Review, Vol. 04, No.1, Nov 2004.
\section*{Abstract}
\addcontentsline{toc}{section}{Abstract}

This paper outlines my concerns with Qualitative Data Analysis' (QDA)
numerous remodelings of Grounded Theory (GT) and the subsequent eroding
impact. I cite several examples of the erosion and summarize essential
elements of classic GT methodology. It is hoped that the article will clarify my
concerns with the continuing enthusiasm but misunderstood embrace of GT by
QDA methodologists and serve as a preliminary guide to novice researchers
who wish to explore the fundamental principles of GT.

\section*{Introduction}

The difference between the particularistic, routine, normative data we all garner
in our everyday lives and scientific data is that a methodology produces the latter.
This is what makes it scientific.
This may sound trite, but it is just the beginning of many complex issues.
Whatever methodology may be chosen to make an ensuing research
scientific has many implicit and explicit problems.
It implies a certain type of data collection,
the pacing, and timing for data collection, a type of analysis,
and a specific type of research product.

In the case of qualitative data, the explicit goal is the description.
The particular issue articulated in much of the literature regarding
qualitative data analysis (QDA) methodology is the data's accuracy,
truth, trustworthiness, or objectivity.
This worrisome accuracy of the data focuses on its subjectivity, interpretative nature,
plausibility, data voice, and constructivism.
Achieving accuracy is always problematic with a QDA methodology.

These are a few of the problems of description. Other QDA problems include
pacing of data collection, the volume of data, the procedure and rigor of data
analysis, generalizability of the unit findings, the framing of the ensuing analysis, and the product. These issues and others are debated at length in the
qualitative research literature. The worrisome accuracy of qualitative data description
continually concerns qualitative researchers and their audiences. I have
addressed these problems at length in "The Grounded Theory Perspective:
Conceptualization Contrasted with Description"
\citep{book.glaser01}.

In this paper, I will take up the conceptual perspective of classic Grounded Theory (GT).
(In some of the research literature,
classic GT methodology has also been termed Glaserian GT,
although I personally prefer the term "classic" as recognition of the methodology’s origins.)
The conceptual nature of classic GT renders it abstract of time, place, and people.
While grounded in data,
the conceptual hypotheses of GT do not entail the problems of accuracy that plague QDA methods.

The mixing of QDA and GT methodologies has the effect of downgrading and eroding the GT goal of conceptual theory.
The result is a default remodeling of classic GT into just another QDA method with its descriptive baggage.
Given the ascending focus on QDA by sheer dint of the number of researchers engaged in qualitative analysis labeled as GT, the apparent merger between the two methodologies results in default remodeling to QDA canons and techniques.
Conceptual requirements of GT methodology are easily lost in QDA problems of accuracy, type data, constructivism, participant voice, data collection rigor according to positivistic representative requirements, however, couched in the flexibility of approach (see \citep{article.lowe97}). 
The result is a blocking of classic GT methodology and the loss of its power to transcend the strictures of worrisome accuracy – 
the prime concern of QDA methods is to produce a conceptual theory that explains fundamental social patterns within the substantive focus of inquiry.

I will address some, but not all, of the myriad of remodeling blocks to classic GT analysis brought on by lacing it with QDA descriptive methodological requirements. 
My goal is to alleviate the bane on good GT analysis brought on by those QDA senior researchers open to no other method, especially the GT method. 
I hope to relieve GT of the excessive scientism brought on by those worried about accuracy and “real” data when creating a scientific product. 
I hope to give explanatory strength to those Ph.D. dissertation level students to stand their GT grounds when struggling in the face of the misapplied QDA critique by their seniors and supervisors.

I wish to remind people, yet again, that classic GT is simply a set of integrated conceptual hypotheses systematically generated to produce an inductive theory about a substantive area. 
Classic GT is a highly structured but eminently flexible methodology. 
Its data collection and analysis procedures are explicit, and the pacing of these procedures is, at once, simultaneous, sequential, subsequent, scheduled, and serendipitous, forming an integrated methodological "whole" that enables the emergence of conceptual theory as distinct from the thematic analysis characteristic of QDA research. 
I have detailed these matters in my books "Theoretical Sensitivity" 
\citep{book.glaser98a},
"Basics of Grounded Theory Analysis" 
\citep{book.glaser92},
"Doing Grounded Theory" 
\citep{book.glaser98b},
and "The Grounded Theory Perspective" 
\citep{book.glaser01}.
Over the years since the initial publication of "Discovery of Grounded Theory" 
\citep{book.glaser67},
the transcendent nature of GT as a general research methodology has been subsumed by the fervent adoption of GT terminology and selective application of discrete aspects of GT methodology into the realm of QDA research methodology. 
This multi-method cherry-picking approach, while obviously
acceptable to QDA, is not compatible with the requirements of GT methodology.

Currently, it appears to be very popular in QDA research substantive and
methodological papers to label QDA as GT for the rhetorical legitimating effect and then to critique its various strategies as somewhat less than possible or effective; then further, to sanctify the mix of methods as one method. 
Classic GT is not what these "adopted QDA" usages would call GT. 
These researchers do not realize that while often using the same type of qualitative data, the GT and QDA methods are sufficiently at odds with each other as to be incapable of integration. 
Each method stands alone as quite legitimate. 
The reader is to keep in mind that this paper is about GT and how to extract it from this remodeling. 
It does not condemn QDA in any way. QDA methods are quite worthy, respectable, and acceptable. 
As I have said above, the choice of methodology to render research representations about qualitative data as scientific is the researcher’s choice. 
But there is a difference between received concepts, problems and
frameworks imposed on data by QDA methods and GT’s focus on the
generation and emergence of concepts, problems, and theoretical codes. 
The choice of methodology should not be confused, lumped, or used piece-meal if GT is involved. 
To do so is to erode the conceptual power of GT.

As such, GT procedures and ideas are used to legitimate and buttress routine QDA methodology. 
Considering the inundation, overwhelming, and overload of QDA dictums, "words," and assumed requirements on GT methodology, the reader will see that it is hard to both assimilate and withstand this avalanche on GT methodology. 
The assault is so strong and well-meaning that many --- particularly novice researchers --- do not know, nor realize, that GT is being remodeled by default.

The view of this paper is that the researcher who has to achieve a GT product to move on with his or her career and skill development is often blocked by the confusion created through this inappropriate mixing of methods and the attendant QDA requirements thus imposed. 
Undoing the blocks to GT by this default remodeling will not be an easy task given the overwhelming confusion that has resulted and seems destined to continue to grow.

I will deal with as many of the blocks as I see relevant but certainly not all. 
If I repeat, it will be from different vantage points to undo QDA remodeling in the service of advancing the GT perspective. 
I will hit hard that GT deals with the data, as it is, not what QDA wishes it to be or, more formally, what QDA preconceives to be accurate and to be forcefully conceptualized. 
This requires honesty about taking all data as it comes, figuring it out, and then its conceptualization. 
I have written at length on "all is data" and on forcing in "Doing Grounded Theory" 
\citep{book.glaser98a}.

As I deal with this escalating remodeling of GT to QDA requirements, my hope is to free GT up to be as originally envisioned. 
In “Theoretical Sensitivity,” I wrote: “The goal of grounded theory is to generate a conceptual theory that accounts for a pattern of behavior which is relevant and problematic for those involved. 
The goal is not voluminous description, nor clever verification”
\citep{book.glaser78}.

\section*{QDA Blocking of GT}

This paper has a simple message. GT is a straightforward methodology. It is a comprehensive, integrated, and highly structured yet eminently flexible process that takes a researcher from the first day in the field to a finished written theory.
Following the full suite of GT procedures based on the constant comparative method, results in a smooth, uninterrupted emergent analysis and the generation of a substantive or formal theory.
When GT procedures are laced with the exhaustive, abundant requirements of QDA methodology, GT becomes distorted, wasting large amounts of precious research time and derailing the knowledge—hence grounding—of GT as to what is really going on. 
The intertwining of GT with preconceived conjecture, preconceptions, forced concepts and organization, logical connections, and before-the-fact professional interest defaults GT to remodeling of GT methodology to the status of a mixed-methods QDA methodology. 
This leads to multiple blocks on conceptual GT.

The word "analysis" is a catchall word for what to do with data.
It is "scientized" up, down, and sideways in QDA methodologies catching up GT analysis in its
wake.
QDA leads to particularistic analysis based on discrete experiences while
blocking the abstract idea of conceptualizing latent patterns upon which GT is
based.
When GT becomes laced with QDA requirements, it is hard to follow to
the point of confusion.
Theory development is confused with QDA description, thereby blocking GT generation of conceptual theory.

GT has clear, extensive procedures.
When brought into QDA, GT abstraction is neglected in favor of accuracy of description
--- the dominant concern of QDA methodology --- 
and GT acquires the QDA problem of worrisome accuracy
---
an irrelevant concern in GT.
To repeat, GT methodology is a straightforward approach to theory generation.
To spend time worrying about its place in QDA methods and science is just fancy, legitimating talk, but the result is the defaulting of GT to the confusion of QDA analysis.

Creswell, in his book “Qualitative Inquiry and Research Design”
\citep{book.creswell98}
lumps GT into comparisons with phenomenology, ethnography, case study, and biographical life history. 
The result of the lumping is a cursory default remodeling of GT to a “kind” of QDA. 
This lumping of GT with other QDA methods prevents GT from standing alone as a transcending general research methodology.
The criteria of Creswell’s continuum organize methods according to when theory is used in research, varying from before the study begins to post-study. 
By study, he means data collection and structuring questions. This is a very weak gradation for discerning the difference between QDA methods and GT methodology. 
Creswell clearly does not discern the difference between generating theory from data collection and generating the theory that applies to the
data once collected. 
Both come during and after data collection but are very differently sourced.
The result is a lumping and confusion of GT with QDA.

Creswell
\citep[p. 86]{book.creswell98}
says:
\begin{quote}
"At the most extreme end of the continuum, toward the 'after' end, I
place grounded theory. 
Strauss and Corbin (1990) are clear that one collects and analyzes data before using theory in a grounded theory study. 
This explains, for example, the women's sexually abuse study by
Morrow and Smith (1995) in which they generate the theory through
data collection, pose it at the end and eschew prescribing a theory at
the beginning of the study. 
In my own studies, I have refrained from
advancing a theory at the beginning of my grounded theory research,
generated the theory through data collection and analysis, posed the
theory as a logic diagram and introduced contending and contrasting
theory with the model I generate at the end of my study 
(Creswell \& Brown 1992, Creswell and Urbom 1997)."
\end{quote}

Creswell may be stating a fundamental tenant of GT --- begin with no
preconceived theory and then generate one during the analysis (unless he meant applying an extant theory). 
As a distinguishing item of GT, however, it is barely a beginning, leaving the reader with no knowledge of how generating is done because the assumption is that it is done by routine QDA. 
Contrasting the generated theory with extant other theories to prove, improve or disprove one or the other neglects or ignores constantly comparing the theories for category and property generation. 
This contrasting with other theories also prevents modifying the GT-generated theory using the other theory as a kind of data.
Both constant comparing and modifying are two vital tenants of GT.

GT may or may not be mentioned in a QDA methodological discussion, 
but its procedures frequently are. 
As such, constant comparative analysis, problem emergence, theoretical sampling, theoretical saturation, conceptual emergence, memoing, sorting, etc. become laced with QDA requirements, thereby defaulting
their rigorous use to a QDA burden. 
This virtual subversion of GT results in complex confusion of an otherwise simple methodology for novice researchers.
The researcher is blocked and no longer freed by the power and autonomy offered by GT to arrive at new emergent, generated theory. 
The ability to be honest about what exactly the data is is consequently distorted by the unattainable quest for QDA accuracy. 
For example, Kathryn MAY unwittingly erodes the GT methodology in QDA fashion when describing the cognitive processes inherent in data analysis.

\begin{quote}
"Doing qualitative research is not a passive endeavor. Despite current
perceptions and student’s prayers, theory does not magically emerge
from data. Nor is it true that, if only one is patient enough, insight
wondrously enlightens the researcher. Rather, data analysis is a
process that requires astute questioning, a relentless search for
answers, active observation, and accurate recall. It is a process of
piecing together data, of making the invisible obvious, of recognizing the
significant from the insignificant, of linking seemingly unrelated facts
logically, of fitting categories one with another, and of attributing
consequences to antecedents. It is a process of conjecture and
verification, of correction and modification, of suggestion and defense. It
is a creative process of organizing data so that the analytic scheme will
appear obvious." 
\citep[p. 10]{incollection.may94}
\end{quote}

Dr. May engages in descriptive capture in QDA fashion and attacks the main tenant of GT, that theory can emerge. 
She is lost inaccurate fact research, which is moot for GT.
She prefers to force the data, making it obey her framework. 
She does not acknowledge the constant comparative method by which theory emerges from all data. 
Again, GT has defaulted to routine QDA.

Similarly, this Ph.D. student—in her e-mail cry to me for help—wanted to do a
GT dissertation but was caught up in QDA and descriptive capture.

\begin{quote}
"I need some guidance. 
I’m on wrong track --- I don't care about the main concerns of clinical social workers in private practice. 
I care about the main concerns of anyone attempting to contextualize practice. 
Maybe the issue is that I'm interested in an activity regardless of the actor. 
If I ask these questions, I have no doubt that main concerns will emerge as well as attempts to continually resolve them. 
This I care about."
(E-mail correspondence, Jan 2002)
\end{quote}

She is caught by the QDA approach to force the data for a professional concern.
She wants to use GT procedures in service of a QDA forcing approach, which defaults GT. 
GT does not work that way, but the prevalence of QDA would have her think that way. 
Later, under my guidance, she let the main concern emerge and did an amazingly good dissertation on binary deconstruction between social worker and client.

The GT problem and core variable must emerge, and it will.
I have seen it hundreds of times. 
Later, when the GT's main concern emerges and is explained in a generated theory, it will have relevance for professional concerns.
Starting before emergence with the professional interest problem is very likely to result in research with little or no relevance in GT --- just routine QDA description with "as if" importance.

Here is a good example of extensive lacing of GT by QDA needs. 
The confusion of QDA requirements and GT procedures, in this example, 
makes it hard to follow and clearly erodes GT by default remodeling.

\begin{quote}
"Comprehension is achieved in grounded theory by using taperecorded,
unstructured interviews and by observing participants in their
daily lives.
However, the assumption of symbolic interactionism that
underlie grounded theory set the stage for examining process, for
identifying stages and phases in the participant’s experience.
Symbolic interaction purports that meaning is socially constructed, negotiated
and changes over time.
Therefore the interview process seeks to elicit a participant’s story,
and this story is told sequentially as the events being reported unfold.
Comprehension is reached when the researcher has
interviewed enough to gain in-depth understanding"
\citep[p. 39]{incollection.morse86}.
\end{quote}

In fact, GT does not require tape-recorded data. 
Field notes are preferable. 
GT uses all types of interviews and, as the study proceeds, the best interview style emerges. 
It is not underlined by symbolic interaction nor constructed data. 
GT uses "all as data," of which these are just one kind of data. 
GT does not preconceive the theoretical code of the process. 
There are over 18 theoretical coding families, of which the process is only one. 
In GT, its relevance must emerge; it is not presumed. 
Interviews lead to many theoretical codes. 
Participant stories are moot. 
Patterns are sought and conceptualized. 
GT does not search for a description of particularistic accounts. 
All data are constantly compared to generate concepts.

Morse continues her description of GT:

\begin{quote}
"Synthesis is facilitated by the adequacy of the data and the processes of
analysis. 
During this phase the researcher is able to create a generalized story and to determine points of departure of variation in this story. 
The process of analysis begins with line-by-line analysis to identify first-level codes. 
Second-level codes are used to identify significant portions of the text and compile these excerpts into categories. 
Writing memos is key to recording insight and facilitates, at an early stage, the development of theory"
\citep[p. 39]{incollection.morse94}.
\end{quote}

It is, indeed, hard to recognize GT procedures in this quote by Morse.
"Adequacy of data" and a "generalized story" smack of worrisome accuracy and descriptive capture, 
which are pure QDA concerns. 
They do not relate to GT procedures. 
GT fractures the story in the service of conceptualization. 
Her approach to line-by-line analysis is a bare reference to the constant comparative process, but that is all. 
Her references to the first level, second-level codes, portions of text, and compiling excerpts into categories are far from the constant comparative method designed to generate conceptual categories and their properties from the outset of data collection and analysis.
Writing memos in GT has to do with the immediate recording of generated theoretical, conceptual ideas grounded in data, not the mystical—perhaps conjectural—insights to which Morse refers.

Morse continues with her description of GT:
\begin{quote}
"As synthesis is gained and the variation in the data becomes evident,
grounded theorists sample according to the theoretical needs of the
study. 
If a negative case is identified, the researcher, theoretically, must
sample for more negative cases until saturation is reached when
synthesis is attained."
\citep[p. 39]{incollection.morse94}.
\end{quote}

Again, finding GT procedures in this description is hard. 
There is always variation in the data. 
GT is concerned with generating a multivariate conceptual theory—not data variation for QDA.
In GT, seeking negative cases is not a procedure. 
This is more likely to be preconceived forcing. 
GT seeks comparative incidents by theoretical sampling. 
The purpose of sampling is to generate categories and their properties. The GT researcher does not know in advance what will be found. 
Incidents sampled may be similar or different, positive or negative.
Morse’s reference to saturation does not imply conceptual saturation; rather, it anticipates simple redundancy without conceptual analysis.

Morse continues:

\begin{quote}
"Theorizing follows from the processes of theoretical sampling.
Typologies are constructed by determining two significant characteristics and sorting participants against each characteristic on a 2×2 matrix.
Diagramming is used to enhance understanding and identifying the
basic social process (BSP) that accounts for most of the variation in the
data."
\citep[p. 39]{incollection.morse94}.
\end{quote}

Theorizing in GT is an emergent process generated by continuous cycling of the integrated processes of collecting, coding, and conceptual analysis with the results written up constantly in memos. 
Theoretical sampling is just one source of grounding during the constant comparative method. 
Preconceiving theoretical codes such as typologies or basic social processes (BSPs) is not GT. 
In GT, relevant theoretical codes emerge in conceptual memo sorting and could be "whatever." 
While the fourfold property space is a good tool, when emergent, for conceptualizing types 
(see "Awareness of Dying" \citep{book.glaser65}).
, it is not for placing or sorting participants, a priori, nor for counting them. 
This is strictly routine, preconceived QDA descriptive capture, not GT.

Morse finishes:

\begin{quote}
"As with the methods previously discussed, recontextualization is determined by the level of abstraction attained in the model development. 
Whereas substantive theory context-bounded, formal theory is more abstract and may be applicable to many settings or other experiences.
\citep[p. 39]{incollection.morse94}."
\end{quote}

This statement is totally wrong for GT, but it addresses the usual QDA quandary of trying to generalize a description of a unit. 
In contrast, GT substantive theory always has general implications and can easily be applied to other substantive areas by the constant comparative method of modifying theory. 
For example, by comparing incidents and modifying the substantive theory of milkmen who engage in cultivating housewives for-profit and recreation, a GT of cultivation can apply easily to doctors cultivating clients to build a practice, thereby expanding the original substantive theory to include cultivating down instead of cultivating up the social scale. 
Formal theory is generated by many such diverse area comparisons done in a concerted way to generate a formal theory of cultivating for recreation, profit, client building, help, donations, etc.

The context must emerge as a relevant category or as a theoretical code like all other categories in a GT. 
It cannot be assumed as relevant in advance. 
As one applies substantive theory elsewhere or generates formal theory, context—when relevant—will emerge.

These quotes clearly lump GT into the multi-method QDA camp, with the result being default remodeling by erosion of classic GT methodology. 
Nowhere does MORSE refer to the GT procedures of delimiting at each phase of generating, of theoretical completeness, conceptual saturation, core variable analysis, open to selective coding, memo banks, analytic rules, theoretical sorting, memo piles writing up, reworking and re-sorting, emergent problem, interchangeability of
indices and theoretical (not substantive) coding. 
The effect of such default remodeling is a great loss of essential GT procedures blocked by the imposition of QDA worrisome accuracy requirements.

GT requires following its rigorous procedures to generate a theory that fits, works, is relevant, and readily modifiable. 
When it is adopted, co-opted, and corrupted by QDA research, a close look at the work often shows that the QDA researcher is tinkering with the GT method. 
He or she brings it into a QDA research design to comply with the strictures and professional expectations of the dominant paradigm. 
Getting some kind of product with a few concepts rescues the QDA research since the QDA description alone does not suffice.
Then, the GT label is used to legitimate the QDA research.

GT stands alone as a conceptual theory-generating methodology. 
It is a general methodology. 
It can use any data, but obviously, the favorite data, to date, is qualitative data. 
Ergo GT is drawn into the QDA multi-method world and eroded by consequence, however unwittingly. 
This revealing of method muddling
(see \citep{article.cynthia92})
of procedures does a tinkering rescue job, but the result is that GT is default remodeled. 
GT becomes considered, wrongly, as an interpretative method, a symbolic interaction method, a constructionist method, a qualitative method, a describing method, a producer of worrisome facts, a memoing method, an interview or field method, and so forth. 
It is clear that this tinkering by QDA researchers indicates they are too derailed by QDA to learn systematic GT procedures. 
At best, a few GT procedures are borrowed out of context.

The above authors are typical of many trying to place GT somewhere in the QDA camp. 
First, they lace it with some QDA requirements and ideas, which
they then use to lump GT into QDA multi-method thought. 
Lumping GT in as a QDA methodology simply does not apply and, indeed, blocks good GT while the default remodeling of GT into another QDA rages on. 
Lumping erodes GT. 
In the remainder of this article, I will try to show how GT stands alone on its own as a conceptualizing methodology. 
My goal will be to bring out the classic GT perspective on how GT analysis is done—to lay this method bare—and in the bargain to show how QDA blocks, as I have said, GT generation and product proof.

\section*{Grounded Theory Procedures}

When not laced and lumped with QDA requirements, GT procedures are fairly simple. 
The blocking problems come with the method mixing. 
I have already written in detail much about GT procedures in "Discovery of Grounded Theory"
\citep{book.glaser67},
"Theoretical Sensitivity", 
\citep{book.glaser78},
"Doing Grounded Theory", 
\citep{book.glaser98a},
"Basics of Grounded Theory Analysis, 
\citep{book.glaser92},
"More Grounded Theory Methodology", 
\citep{book.glaser94},
and 
"The Grounded Theory Perspective" 
\citep{book.glaser01},
 all by Sociology Press. 
I have also published many examples of a "good" GT analysis
---" Examples of Grounded Theory"
\citep{book.glaser93},
"Grounded Theory 1984 to 1994"
\citep{book.glaser95},
"Gerund Grounded Theory"
\citep{book.glaser98b},
--- and have given many references in my books.

The GT product is simple. 
It is not a factual description. 
It is a set of carefully grounded concepts organized around a core category and integrated into hypotheses. 
The generated theory explains the preponderance of behavior in a substantive area, with the prime mover of this behavior surfacing as the main concern of the primary participants. 
I have said over and over that GT is not findings, not accurate facts, and not description. 
It is just a straightforward conceptualization integrated into theory --- a set of plausible, grounded hypotheses. 
It is just that --- no more --- and it is readily modifiable as new data come from whatever source --- literature, new data, collegial comments, etc. 
The constant comparative method weaves the new data into the sub-conceptualization.
What is important is to use the complete package of GT procedures as an
integrated methodological whole.

The following summarizes the essential elements of GT methodology:
Bear in mind, when reading this summary that the goal of GT is conceptual theory abstract of time, place, and people. 
The goal of GT is NOT the QDA quest for accurate description.

\section*{Theoretical sensitivity}

The ability to generate concepts from data and to relate them according to normal models of theory in general, and theory development in sociology in particular, is the essence of theoretical sensitivity. 
Generating a theory from data means that most hypotheses and concepts come from the data and are systematically worked out with the data during the research. 
A researcher requires two essential characteristics for the development of theoretical sensitivity. 
First, he or she must have the personal and temperamental bent to maintain analytic distance, tolerate confusion and regression while remaining open, trusting to preconscious processing and to
conceptual emergence. 
Second, he/she must have the ability to develop theoretical insight into the area of research combined with the ability to make something of these insights. 
He/she must have the ability to conceptualize and organize, make abstract connections, visualize and think multivariate. 
The first step in gaining theoretical sensitivity is to enter the research setting with as few predetermined ideas as possible
--- especially logically deducted, a prior hypothesis. 
The research problem and its delimitation are discovered. 
The pre-framework efforts of QDA --- block this theoretical sensitivity.

\section*{Getting started}

A good GT analysis starts right off with regular daily data collecting, coding, and analysis. 
The start is not blocked by a preconceived problem, a methods chapter
or a literature review. 
The focus and flow are immediately into conceptualization using the constant comparative method. 
The best way to do GT is to just do it. 
It cannot fail as the social-psychological world of structure, culture, social
interaction, social organization, etc., goes on irrespective. 
There always is the main concern, and there always is a prime mover. 
As an open, generative, and emergent methodology, 
GT provides an honest approach to the data that lets the natural organization of substantive life emerge. 
The GT researcher listens to participants venting issues rather than encouraging them to talk about a subject of little interest. 
The mandate is to remain open to what is happening and not to start filtering data through preconceived hypotheses and biases to listen and observe and thereby discover the main concern of the participants in the field and how they resolve this concern. 
The forcing, preconceived notions of an initial professional problem, 
or an extant theory and framework are suspended in the service of seeing what will emerge conceptually by constant comparative analysis. 
When QDA requires this preconception, GT is rendered non-emergent through coding and memoing as the researcher tries to follow a non-emergent problem.

\section*{All is data}

GT stands alone as a conceptual theory-generating methodology. 
It can use any data, but obviously, the favorite data to date is qualitative. 
While interviews are the most popular, 
GT works with any data -- "all is data" --- not just one specific data. 
It is up to the GT researcher to figure out what data they are getting. 
The data may be baseline, vague, interpreted, or proper-line. 
The data is not to be discounted as "not objective," 
as "subjective," "obvious," "constructed," etc., as we find in QDA critiques. 
There is always a perception of perception as the conceptual level rises.
We are all stuck with a "human" view of what is going on
and hazy concepts and descriptions about it. 
GT procedures sharpen the generated concepts systematically.

\section*{Use of the literature}

It is critical in GT methodology to avoid unduly influencing the pre-conceptualization of the research through extensive reading in the substantive area and the forcing of extant theoretical overlays on the collection and analysis of data. 
To undertake an extensive review of the literature before the emergence of a core category violates the basic premise of GT --- that being, the theory emerges from the data, not from extant theory. 
It also runs the risk of clouding the researcher’s ability to remain open to the emergence of a completely new core category that has not figured prominently in the research to date, thereby thwarting the theoretical sensitivity. 
Practically, it may well result in the researcher spending valuable time on an area of literature that proves to be of
little significance to the resultant GT. 
Instead, GT methodology treats the literature as another source of data to be integrated into the constant comparative analysis process once the core category, its properties, and related categories have emerged, and the basic conceptual development is well underway. 
The pre study literature review of QDA is a waste of time and a
derailing of relevance for the GT Study.

\section*{Theoretical coding}

The conceptualization of data through coding is the foundation of GT development. 
Incidents articulated in the data are analyzed and coded --- using the constant comparative method --- to generate initially substantive and later theoretical categories. 
The essential relationship between data and theory is a conceptual code. The code conceptualizes the underlying pattern of a set of empirical indicators within the data. 
Coding gets the analyst off the empirical level by fracturing the data, then conceptually grouping it into codes that then become the theory that explains what is happening in the data. 
A code gives the researcher a condensed, abstract view with a scope of the data that includes otherwise seemingly disparate phenomenon. 
Substantive codes conceptualize the empirical substance of the area of research. 
Theoretical codes conceptualize how the substantive codes may relate to each other as hypotheses to be integrated into the theory. 
Theoretical codes give integrative scope, broad pictures, and a new perspective. 
They help the analyst maintain the conceptual level in writing about concepts and their interrelations.

\section*{Open coding}

In the beginning, with open coding --- and, a minimum of the preconception --- the analyst, is most tested as to his trust in himself, in the grounded method, and in the skill to use the method and as to the ability to generate codes and find relevance. 
The process begins with line-by-line open coding of the data to
identify substantive codes emergent within the data. 
The analyst begins by coding the data in every way possible --- "running the data open."
From the start, the analyst asks a set of questions of the data --- "What is this data a study of?"
"What category does this incident indicate?"
"What is actually happening in the data?" 
"What is the main concern being faced by the participants?" 
and 
"What accounts for the continual resolving of this concern?" 
These questions keep the analyst theoretically sensitive and transcending when analyzing, collecting, and coding the data. 
They force him/her to focus on patterns among incidents that yield codes and rise conceptually above detailed descriptions of incidents. 
The analyst codes for as many categories as fit successive, different incidents while coding into as many categories as possible. 
New categories emerge, and new incidents fit into existing categories.












\vspace{100pt}

Conceptual specification, not the definition, is the focus of GT. 
The GT concept indicator model requires concepts and their dimensions to earn their way into the theory by a systematic generation of data. 
Changing incidents (indicators) and thereby generating new properties of a code can only go so far before the analyst discovers saturation of ideas through the interchangeability of indicators.
This interchangeability produces, at the same time, the transferability of the theory to other areas by linking to incidents (indicators) in other substantive or sub-substantive areas that produce the same category or properties of it.
Interchangeability produces saturation of concepts and their properties, not redundancy of description as some QDA methodologists would have it 
(see \citep{article.morse95}).
\vspace{100pt}

\begin{quote}
''Today’s general textbooks perpetuate the established marketing
management epic from the 1960s with the new just added as extras. It
is further my contention that marketing education has taken an
unfortunate direction and has crossed the fine line between education
and brainwashing. The countdown of a painful—but revitalizing—
process of deprogramming has to be initiated.

What do we need in such a situation? A shrink? No, it is less
sophisticated than that. All we need is systematic application of
common sense, both in academe and in corporations.We need to use
our observational capacity in an inductive mode and allow it to receive
the true story of life, search for patterns and build theory.Yes, theory.
General marketing theory that helps us put events and activities into a
context. This is all within the spirit of grounded theory, wide spread in
sociology but little understood by marketers. My interpretation of a
recent book on the subject by 
\citep{book.glaser01}
is as follows: ‘take the
elevator from the ground floor of raw substantive data and description to
the penthouse of conceptualization and general theory. And do this
without paying homage to the legacy of extant theory.’ In doing this,
complexity, fuzziness and ambiguity are received with cheers by the
researchers and not shunned as unorderly and threatening as they are
by quantitative researchers. Good theory is useful for scholars and
practicing managers alike \citep[p. 132]{article.gummensson02}.''
\end{quote}

\section*{Copyright}
\addcontentsline{toc}{section}{Copyright}

As an academic, open access journal,
the Grounded Theory Review is dedicated to the open exchange of information.
Copies of articles in this journal may be distributed for noncommercial purposes
free of charge and without permission.
The copyright of papers published in the Grounded Theory Review stay with the author.
The Grounded Theory Review does not have a specific creative commons license,
but allows readers to read, download, copy, distribute, print, search,
or link to the full texts of its articles.  Derivative works are allowed with proper citation.


\bibliographystyle{apalikerd}
\bibliography{bib}

