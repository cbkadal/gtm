% REV01 Thu 05 Aug 2021 14:26:40 WIB
% START Tue 03 Aug 2021 13:44:15 WIB

\chapter{Remodeling Grounded Theory}

\section*{Abstract}
\addcontentsline{toc}{section}{Abstract}

This paper outlines my concerns with Qualitative Data Analysis (QDA)
numerous remodelings of Grounded Theory (GT) and the subsequent eroding
impact. I cite several examples of the erosion and summarize essential
elements of classic GT methodology. It is hoped that the article will clarify my
concerns with the continuing enthusiasm but misunderstood embrace of GT by
QDA methodologists and serve as a preliminary guide to novice researchers
who wish to explore the fundamental principles of GT.

\begin{quote}
''Today’s general textbooks perpetuate the established marketing
management epic from the 1960s with the new just added as extras. It
is further my contention that marketing education has taken an
unfortunate direction and has crossed the fine line between education
and brainwashing. The countdown of a painful—but revitalizing—
process of deprogramming has to be initiated.

What do we need in such a situation? A shrink? No, it is less
sophisticated than that. All we need is systematic application of
common sense, both in academe and in corporations.We need to use
our observational capacity in an inductive mode and allow it to receive
the true story of life, search for patterns and build theory.Yes, theory.
General marketing theory that helps us put events and activities into a
context. This is all within the spirit of grounded theory, wide spread in
sociology but little understood by marketers. My interpretation of a
recent book on the subject by 
\citep{book.glaser01}
is as follows: ‘take the
elevator from the ground floor of raw substantive data and description to
the penthouse of conceptualization and general theory. And do this
without paying homage to the legacy of extant theory.’ In doing this,
complexity, fuzziness and ambiguity are received with cheers by the
researchers and not shunned as unorderly and threatening as they are
by quantitative researchers. Good theory is useful for scholars and
practicing managers alike \citep[p. 132]{article.gummensson02}.''
\end{quote}

\nocite{article.cynthia10}.
\bibliographystyle{apalikerd}
\bibliography{bib}

